% Options for packages loaded elsewhere
\PassOptionsToPackage{unicode}{hyperref}
\PassOptionsToPackage{hyphens}{url}
%
\documentclass[
]{article}
\usepackage{amsmath,amssymb}
\usepackage{iftex}
\ifPDFTeX
  \usepackage[T1]{fontenc}
  \usepackage[utf8]{inputenc}
  \usepackage{textcomp} % provide euro and other symbols
\else % if luatex or xetex
  \usepackage{unicode-math} % this also loads fontspec
  \defaultfontfeatures{Scale=MatchLowercase}
  \defaultfontfeatures[\rmfamily]{Ligatures=TeX,Scale=1}
\fi
\usepackage{lmodern}
\ifPDFTeX\else
  % xetex/luatex font selection
\fi
% Use upquote if available, for straight quotes in verbatim environments
\IfFileExists{upquote.sty}{\usepackage{upquote}}{}
\IfFileExists{microtype.sty}{% use microtype if available
  \usepackage[]{microtype}
  \UseMicrotypeSet[protrusion]{basicmath} % disable protrusion for tt fonts
}{}
\makeatletter
\@ifundefined{KOMAClassName}{% if non-KOMA class
  \IfFileExists{parskip.sty}{%
    \usepackage{parskip}
  }{% else
    \setlength{\parindent}{0pt}
    \setlength{\parskip}{6pt plus 2pt minus 1pt}}
}{% if KOMA class
  \KOMAoptions{parskip=half}}
\makeatother
\usepackage{xcolor}
\usepackage[margin=1in]{geometry}
\usepackage{color}
\usepackage{fancyvrb}
\newcommand{\VerbBar}{|}
\newcommand{\VERB}{\Verb[commandchars=\\\{\}]}
\DefineVerbatimEnvironment{Highlighting}{Verbatim}{commandchars=\\\{\}}
% Add ',fontsize=\small' for more characters per line
\usepackage{framed}
\definecolor{shadecolor}{RGB}{248,248,248}
\newenvironment{Shaded}{\begin{snugshade}}{\end{snugshade}}
\newcommand{\AlertTok}[1]{\textcolor[rgb]{0.94,0.16,0.16}{#1}}
\newcommand{\AnnotationTok}[1]{\textcolor[rgb]{0.56,0.35,0.01}{\textbf{\textit{#1}}}}
\newcommand{\AttributeTok}[1]{\textcolor[rgb]{0.13,0.29,0.53}{#1}}
\newcommand{\BaseNTok}[1]{\textcolor[rgb]{0.00,0.00,0.81}{#1}}
\newcommand{\BuiltInTok}[1]{#1}
\newcommand{\CharTok}[1]{\textcolor[rgb]{0.31,0.60,0.02}{#1}}
\newcommand{\CommentTok}[1]{\textcolor[rgb]{0.56,0.35,0.01}{\textit{#1}}}
\newcommand{\CommentVarTok}[1]{\textcolor[rgb]{0.56,0.35,0.01}{\textbf{\textit{#1}}}}
\newcommand{\ConstantTok}[1]{\textcolor[rgb]{0.56,0.35,0.01}{#1}}
\newcommand{\ControlFlowTok}[1]{\textcolor[rgb]{0.13,0.29,0.53}{\textbf{#1}}}
\newcommand{\DataTypeTok}[1]{\textcolor[rgb]{0.13,0.29,0.53}{#1}}
\newcommand{\DecValTok}[1]{\textcolor[rgb]{0.00,0.00,0.81}{#1}}
\newcommand{\DocumentationTok}[1]{\textcolor[rgb]{0.56,0.35,0.01}{\textbf{\textit{#1}}}}
\newcommand{\ErrorTok}[1]{\textcolor[rgb]{0.64,0.00,0.00}{\textbf{#1}}}
\newcommand{\ExtensionTok}[1]{#1}
\newcommand{\FloatTok}[1]{\textcolor[rgb]{0.00,0.00,0.81}{#1}}
\newcommand{\FunctionTok}[1]{\textcolor[rgb]{0.13,0.29,0.53}{\textbf{#1}}}
\newcommand{\ImportTok}[1]{#1}
\newcommand{\InformationTok}[1]{\textcolor[rgb]{0.56,0.35,0.01}{\textbf{\textit{#1}}}}
\newcommand{\KeywordTok}[1]{\textcolor[rgb]{0.13,0.29,0.53}{\textbf{#1}}}
\newcommand{\NormalTok}[1]{#1}
\newcommand{\OperatorTok}[1]{\textcolor[rgb]{0.81,0.36,0.00}{\textbf{#1}}}
\newcommand{\OtherTok}[1]{\textcolor[rgb]{0.56,0.35,0.01}{#1}}
\newcommand{\PreprocessorTok}[1]{\textcolor[rgb]{0.56,0.35,0.01}{\textit{#1}}}
\newcommand{\RegionMarkerTok}[1]{#1}
\newcommand{\SpecialCharTok}[1]{\textcolor[rgb]{0.81,0.36,0.00}{\textbf{#1}}}
\newcommand{\SpecialStringTok}[1]{\textcolor[rgb]{0.31,0.60,0.02}{#1}}
\newcommand{\StringTok}[1]{\textcolor[rgb]{0.31,0.60,0.02}{#1}}
\newcommand{\VariableTok}[1]{\textcolor[rgb]{0.00,0.00,0.00}{#1}}
\newcommand{\VerbatimStringTok}[1]{\textcolor[rgb]{0.31,0.60,0.02}{#1}}
\newcommand{\WarningTok}[1]{\textcolor[rgb]{0.56,0.35,0.01}{\textbf{\textit{#1}}}}
\usepackage{graphicx}
\makeatletter
\def\maxwidth{\ifdim\Gin@nat@width>\linewidth\linewidth\else\Gin@nat@width\fi}
\def\maxheight{\ifdim\Gin@nat@height>\textheight\textheight\else\Gin@nat@height\fi}
\makeatother
% Scale images if necessary, so that they will not overflow the page
% margins by default, and it is still possible to overwrite the defaults
% using explicit options in \includegraphics[width, height, ...]{}
\setkeys{Gin}{width=\maxwidth,height=\maxheight,keepaspectratio}
% Set default figure placement to htbp
\makeatletter
\def\fps@figure{htbp}
\makeatother
\setlength{\emergencystretch}{3em} % prevent overfull lines
\providecommand{\tightlist}{%
  \setlength{\itemsep}{0pt}\setlength{\parskip}{0pt}}
\setcounter{secnumdepth}{-\maxdimen} % remove section numbering
\ifLuaTeX
  \usepackage{selnolig}  % disable illegal ligatures
\fi
\IfFileExists{bookmark.sty}{\usepackage{bookmark}}{\usepackage{hyperref}}
\IfFileExists{xurl.sty}{\usepackage{xurl}}{} % add URL line breaks if available
\urlstyle{same}
\hypersetup{
  pdftitle={Tutorial\_3},
  pdfauthor={Andrea B},
  hidelinks,
  pdfcreator={LaTeX via pandoc}}

\title{Tutorial\_3}
\author{Andrea B}
\date{2023-10-05}

\begin{document}
\maketitle

\hypertarget{uxe1lgebra-lineal-en-matrices-de-n-dimensiones.}{%
\section{\texorpdfstring{\textbf{Álgebra lineal en matrices de n
dimensiones.}}{Álgebra lineal en matrices de n dimensiones.}}\label{uxe1lgebra-lineal-en-matrices-de-n-dimensiones.}}

En este tutorial, se utilizará una descomposición matricial del álgebra
lineal, la descomposición de valores singulares, para generar una
aproximación comprimida de una imagen.Se usará la imagen del módulo
scipy.misc:

\begin{Shaded}
\begin{Highlighting}[]
\ImportTok{import}\NormalTok{ scipy.datasets}
\NormalTok{img}\OperatorTok{=}\NormalTok{ scipy.datasets.face()}
\end{Highlighting}
\end{Shaded}

\begin{Shaded}
\begin{Highlighting}[]
\ImportTok{import}\NormalTok{ numpy }\ImportTok{as}\NormalTok{ np}
\BuiltInTok{type}\NormalTok{(img)}
\end{Highlighting}
\end{Shaded}

\begin{verbatim}
## <class 'numpy.ndarray'>
\end{verbatim}

Podemos ver la imagen usando la función y el comando especial
Python,~para mostrar gráficos en línea:

\begin{Shaded}
\begin{Highlighting}[]
\ImportTok{import}\NormalTok{ matplotlib.pyplot }\ImportTok{as}\NormalTok{ plt}
\NormalTok{plt.imshow(img)}
\NormalTok{plt.show()}
\end{Highlighting}
\end{Shaded}

\includegraphics{Tutorial_3_files/figure-latex/unnamed-chunk-3-1.pdf}

\begin{Shaded}
\begin{Highlighting}[]
\NormalTok{plt.imshow(img)}
\NormalTok{plt.show()}
\end{Highlighting}
\end{Shaded}

\includegraphics{Tutorial_3_files/figure-latex/unnamed-chunk-4-3.pdf}

\hypertarget{propiedades-de-forma-eje-y-matriz}{%
\subsubsection{Propiedades de forma, eje y
matriz}\label{propiedades-de-forma-eje-y-matriz}}

Tenga en cuenta que, en álgebra lineal, la dimensión de un vector se
refiere al número de entradas en una matriz. Primero, verificamos la
forma de los datos en la matriz. Dado que esta imagen es bidimensional
(los píxeles de la imagen forman un rectángulo), podríamos esperar que
una matriz bidimensional la represente (una matriz). Sin embargo, usar
la shape propiedad de esta matriz NumPy nos da un resultado diferente.

\begin{Shaded}
\begin{Highlighting}[]
\NormalTok{img.shape}
\end{Highlighting}
\end{Shaded}

\begin{verbatim}
## (768, 1024, 3)
\end{verbatim}

La salida es una tupla con tres elementos, lo que significa que se trata
de una matriz tridimensional. De hecho, como se trata de una imagen en
color y hemos utilizado la imread función para leerla, los datos están
organizados en tres matrices 2D, que representan canales de color (en
este caso, rojo, verde y azul - RGB). Puedes ver esto mirando la forma
de arriba: indica que tenemos una matriz de 3 matrices, cada una con una
forma de 768x1024.

Además, usando la ndim propiedad de esta matriz, podemos ver que:

\begin{Shaded}
\begin{Highlighting}[]
\NormalTok{img.ndim}
\end{Highlighting}
\end{Shaded}

\begin{verbatim}
## 3
\end{verbatim}

NumPy se refiere a cada dimensión como un~\emph{eje}~.~Por cómo~imread
funciona, el~\emph{primer índice en el tercer eje}~son los datos del
píxel rojo de nuestra imagen.~Podemos acceder a esto usando la sintaxis

\begin{Shaded}
\begin{Highlighting}[]
\NormalTok{img[:, :, }\DecValTok{0}\NormalTok{]}
\end{Highlighting}
\end{Shaded}

\begin{verbatim}
## array([[121, 138, 153, ..., 119, 131, 139],
##        [ 89, 110, 130, ..., 118, 134, 146],
##        [ 73,  94, 115, ..., 117, 133, 144],
##        ...,
##        [ 87,  94, 107, ..., 120, 119, 119],
##        [ 85,  95, 112, ..., 121, 120, 120],
##        [ 85,  97, 111, ..., 120, 119, 118]], dtype=uint8)
\end{verbatim}

En el resultado anterior, podemos ver que cada valor es un valor entero
entre 0 y 255, que representa el nivel de rojo en cada píxel de la
imagen correspondiente.Como era de esperar, esta es una matriz de
768x1024:

\begin{Shaded}
\begin{Highlighting}[]
\NormalTok{img[:, :, }\DecValTok{0}\NormalTok{].shape}
\end{Highlighting}
\end{Shaded}

\begin{verbatim}
## (768, 1024)
\end{verbatim}

Dado que vamos a realizar operaciones de álgebra lineal con estos datos,
podría ser más interesante tener números reales entre 0 y 1 en cada
entrada de las matrices para representar los valores RGB.~Podemos
hacerlo configurando.

\begin{Shaded}
\begin{Highlighting}[]
\NormalTok{img\_array }\OperatorTok{=}\NormalTok{ img }\OperatorTok{/} \DecValTok{255}
\end{Highlighting}
\end{Shaded}

Esta operación, que divide una matriz por un escalar, funciona gracias a
las reglas de transmisión de NumPy .Puede comprobar que lo anterior
funciona haciendo algunas pruebas; por ejemplo, consultando sobre los
valores máximos y mínimos para esta matriz:

\begin{Shaded}
\begin{Highlighting}[]
\NormalTok{img\_array.}\BuiltInTok{max}\NormalTok{(), img\_array.}\BuiltInTok{min}\NormalTok{()}
\end{Highlighting}
\end{Shaded}

\begin{verbatim}
## (1.0, 0.0)
\end{verbatim}

o comprobando el tipo de datos en la matriz:

\begin{Shaded}
\begin{Highlighting}[]
\NormalTok{img\_array.dtype}
\end{Highlighting}
\end{Shaded}

\begin{verbatim}
## dtype('float64')
\end{verbatim}

Tenga en cuenta que podemos asignar cada canal de color a una matriz
separada usando la sintaxis de corte:

\begin{Shaded}
\begin{Highlighting}[]
\NormalTok{red\_array }\OperatorTok{=}\NormalTok{ img\_array[:, :, }\DecValTok{0}\NormalTok{]}
\NormalTok{green\_array }\OperatorTok{=}\NormalTok{ img\_array[:, :, }\DecValTok{1}\NormalTok{]}
\NormalTok{blue\_array }\OperatorTok{=}\NormalTok{ img\_array[:, :, }\DecValTok{2}\NormalTok{]}
\end{Highlighting}
\end{Shaded}

\hypertarget{operaciones-sobre-un-eje}{%
\subsubsection{Operaciones sobre un
eje}\label{operaciones-sobre-un-eje}}

Es posible utilizar métodos del álgebra lineal para aproximar un
conjunto de datos existente.~Aquí, usaremos la svd (descomposición de
valores singulares) para intentar reconstruir una imagen que utiliza
menos información de valores singulares que la original, conservando al
mismo tiempo algunas de sus características.

Para continuar, importe el submódulo de álgebra lineal desde NumPy:

\begin{Shaded}
\begin{Highlighting}[]
\ImportTok{from}\NormalTok{ numpy }\ImportTok{import}\NormalTok{ linalg}
\end{Highlighting}
\end{Shaded}

Para extraer información de una matriz determinada, podemos usar el SVD
para obtener 3 matrices que se pueden multiplicar para obtener la matriz
original.De la teoría del álgebra lineal, dada una matriz A se puede
calcular el siguiente producto:

Observe que podemos usar el~@ operador (el operador de multiplicación de
matrices para matrices NumPy para esto:

\begin{Shaded}
\begin{Highlighting}[]
\NormalTok{img\_gray }\OperatorTok{=}\NormalTok{ img\_array }\OperatorTok{@}\NormalTok{ [}\FloatTok{0.2126}\NormalTok{, }\FloatTok{0.7152}\NormalTok{, }\FloatTok{0.0722}\NormalTok{]}
\end{Highlighting}
\end{Shaded}

Ahora img\_gray tiene forma.

\begin{Shaded}
\begin{Highlighting}[]
\NormalTok{img\_gray.shape}
\end{Highlighting}
\end{Shaded}

\begin{verbatim}
## (768, 1024)
\end{verbatim}

Para ver si esto tiene sentido en nuestra imagen, debemos usar un mapa
de colores matplotlib correspondiente al color que deseamos ver en
nuestra imagen (de lo contrario,matplotlib usaremos por defecto un mapa
de colores que no corresponde a los datos reales).

En nuestro caso, nos estamos aproximando a la parte en escala de grises
de la imagen, por lo que usaremos el mapa de colores gray:

\begin{Shaded}
\begin{Highlighting}[]
\NormalTok{plt.imshow(img\_gray, cmap}\OperatorTok{=}\StringTok{"gray"}\NormalTok{)}
\NormalTok{plt.show()}
\end{Highlighting}
\end{Shaded}

\includegraphics{Tutorial_3_files/figure-latex/unnamed-chunk-16-5.pdf}

Ahora, aplicando la función linalg.svd a esta matriz, obtenemos la
siguiente descomposición:

\begin{Shaded}
\begin{Highlighting}[]
\NormalTok{U, s, Vt }\OperatorTok{=}\NormalTok{ linalg.svd(img\_gray)}
\end{Highlighting}
\end{Shaded}

Comprobemos que esto es lo que esperábamos:

\begin{Shaded}
\begin{Highlighting}[]
\NormalTok{U.shape, s.shape, Vt.shape}
\end{Highlighting}
\end{Shaded}

\begin{verbatim}
## ((768, 768), (768,), (1024, 1024))
\end{verbatim}

Da como resultado un valueError.~Esto sucede porque tener una matriz
unidimensional para S, en este caso, es mucho más económico en la
práctica que construir una matriz diagonal con los mismos datos.~Para
reconstruir la matriz original, podemos reconstruir la matriz
diagonal.Σcon los elementos de S en su diagonal y con las dimensiones
adecuadas para multiplicar: en nuestro caso, Σ debe ser 768x1024 ya que
u es 768x768 y vt es 1024x1024.~Para sumar los valores singulares a la
diagonal de sigma, usaremos la~función:

\begin{Shaded}
\begin{Highlighting}[]
\ImportTok{import}\NormalTok{ numpy }\ImportTok{as}\NormalTok{ np}
\NormalTok{Sigma }\OperatorTok{=}\NormalTok{ np.zeros((U.shape[}\DecValTok{1}\NormalTok{], Vt.shape[}\DecValTok{0}\NormalTok{]))}
\NormalTok{np.fill\_diagonal(Sigma, s)}
\end{Highlighting}
\end{Shaded}

\hypertarget{aproximaciuxf3n}{%
\subsection{\texorpdfstring{\textbf{Aproximación}}{Aproximación}}\label{aproximaciuxf3n}}

El módulo linalg incluye una norm función que calcula la norma de un
vector o matriz representado en una matriz NumPy.~Por ejemplo, a partir
de la explicación anterior de SVD, esperaríamos que la norma de la
diferencia entre img\_gray el producto SVD reconstruido y el producto
SVD fuera pequeña.~Como era de esperar, deberías ver algo como

\begin{Shaded}
\begin{Highlighting}[]
\NormalTok{linalg.norm(img\_gray }\OperatorTok{{-}}\NormalTok{ U }\OperatorTok{@}\NormalTok{ Sigma }\OperatorTok{@}\NormalTok{ Vt)}
\end{Highlighting}
\end{Shaded}

\begin{verbatim}
## 1.5985090441460665e-12
\end{verbatim}

También podríamos haber usado la función numpy.allclose para asegurarnos
de que el producto reconstruido esté, de hecho,~\emph{cerca}~de nuestra
matriz original (la diferencia entre las dos matrices es pequeña):

\begin{Shaded}
\begin{Highlighting}[]
\NormalTok{np.allclose(img\_gray, U }\OperatorTok{@}\NormalTok{ Sigma }\OperatorTok{@}\NormalTok{ Vt)}
\end{Highlighting}
\end{Shaded}

\begin{verbatim}
## True
\end{verbatim}

Para ver si una aproximación es razonable, podemos comprobar los valores
en S:

\begin{Shaded}
\begin{Highlighting}[]
\NormalTok{plt.plot(s)}
\NormalTok{plt.show()}
\end{Highlighting}
\end{Shaded}

\includegraphics{Tutorial_3_files/figure-latex/unnamed-chunk-22-7.pdf}

En el gráfico, podemos ver que aunque tenemos 768 valores singulares en
S, la mayoría de ellos (después de la entrada número 150
aproximadamente) son bastante pequeños. Por lo tanto, podría tener
sentido utilizar sólo la información relacionada con los primeros
(digamos, 50) \emph{valores singulares} para construir una aproximación
más económica a nuestra imagen.

La idea es considerar todos los k valores singulares excepto los
primeros en sigma (que son los mismos que en s) como ceros, manteniendo
u e vt intacto, y calcular el producto de estas matrices como
aproximación.

Por ejemplo, si elegimos

\begin{Shaded}
\begin{Highlighting}[]
\NormalTok{k }\OperatorTok{=} \DecValTok{10}
\end{Highlighting}
\end{Shaded}

podemos construir la aproximación haciendo

\begin{Shaded}
\begin{Highlighting}[]
\NormalTok{approx }\OperatorTok{=}\NormalTok{ U }\OperatorTok{@}\NormalTok{ Sigma[:, :k] }\OperatorTok{@}\NormalTok{ Vt[:k, :]}
\end{Highlighting}
\end{Shaded}

Tenga en cuenta que tuvimos que usar solo las primeras k filas de vt, ya
que todas las demás filas se multiplicarían por los ceros
correspondientes a los valores singulares que eliminamos de esta
aproximación.

\begin{Shaded}
\begin{Highlighting}[]
\NormalTok{plt.imshow(approx, cmap}\OperatorTok{=}\StringTok{"gray"}\NormalTok{)}
\NormalTok{plt.show()}
\end{Highlighting}
\end{Shaded}

\includegraphics{Tutorial_3_files/figure-latex/unnamed-chunk-25-9.pdf}

\hypertarget{aplicando-a-todos-los-colores}{%
\subsubsection{Aplicando a todos los
colores}\label{aplicando-a-todos-los-colores}}

Ahora queremos hacer el mismo tipo de operación, pero con los tres
colores. Nuestro primer instinto podría ser repetir la misma operación
que hicimos anteriormente para cada matriz de color individualmente.
\emph{Sin embargo, la transmisión} de NumPy se encarga de esto por
nosotros.

Si nuestra matriz tiene más de dos dimensiones, entonces el SVD se puede
aplicar a todos los ejes a la vez. Sin embargo, las funciones de álgebra
lineal en NumPy esperan ver una matriz de la forma , donde el primer eje
representa el número de matrices en la pila. En nuestro caso,

\begin{Shaded}
\begin{Highlighting}[]
\NormalTok{img\_array.shape}
\end{Highlighting}
\end{Shaded}

\begin{verbatim}
## (768, 1024, 3)
\end{verbatim}

indica que el eje se reordenará de manera que la forma final de la
matriz transpuesta se reordenará de acuerdo con los índices .Veamos cómo
funciona esto para nuestra matriz:

\begin{Shaded}
\begin{Highlighting}[]
\NormalTok{img\_array\_transposed }\OperatorTok{=}\NormalTok{ np.transpose(img\_array, (}\DecValTok{2}\NormalTok{, }\DecValTok{0}\NormalTok{, }\DecValTok{1}\NormalTok{))}
\NormalTok{img\_array\_transposed.shape}
\end{Highlighting}
\end{Shaded}

\begin{verbatim}
## (3, 768, 1024)
\end{verbatim}

Ahora estamos listos para aplicar la SVD:

\begin{Shaded}
\begin{Highlighting}[]
\NormalTok{U, s, Vt }\OperatorTok{=}\NormalTok{ linalg.svd(img\_array\_transposed)}
\end{Highlighting}
\end{Shaded}

Finalmente, para obtener la imagen aproximada completa, necesitamos
volver a ensamblar estas matrices en la aproximación.~Ahora, tenga en
cuenta que

\begin{Shaded}
\begin{Highlighting}[]
\NormalTok{U.shape, s.shape, Vt.shape}
\end{Highlighting}
\end{Shaded}

\begin{verbatim}
## ((3, 768, 768), (3, 768), (3, 1024, 1024))
\end{verbatim}

\hypertarget{productos-con-matrices-n-dimensionales}{%
\subsubsection{Productos con matrices
n-dimensionales}\label{productos-con-matrices-n-dimensionales}}

Si ha trabajado antes con matrices de una o dos dimensiones en NumPy,
puede usar numpy (o el \texttt{@}operador) indistintamente. Sin embargo,
para matrices de n dimensiones, funcionan de maneras muy diferentes.

Ahora, para construir nuestra aproximación, primero debemos asegurarnos
de que nuestros valores singulares estén listos para la multiplicación,
por lo que construimos nuestra matriz de manera similar a lo que hicimos
antes. La matriz debe tener dimensiones . Para sumar los valores
singulares a la diagonal de , usaremos nuevamente la función , usando
cada una de las 3 filas como diagonal para cada una de las 3 matrices en
:

\begin{Shaded}
\begin{Highlighting}[]
\NormalTok{Sigma }\OperatorTok{=}\NormalTok{ np.zeros((}\DecValTok{3}\NormalTok{, }\DecValTok{768}\NormalTok{, }\DecValTok{1024}\NormalTok{))}
\ControlFlowTok{for}\NormalTok{ j }\KeywordTok{in} \BuiltInTok{range}\NormalTok{(}\DecValTok{3}\NormalTok{):}
\NormalTok{    np.fill\_diagonal(Sigma[j, :, :], s[j, :])}
\end{Highlighting}
\end{Shaded}

Ahora, si deseamos reconstruir el SVD completo (sin aproximación),
podemos hacer

\begin{Shaded}
\begin{Highlighting}[]
\NormalTok{reconstructed }\OperatorTok{=}\NormalTok{ U }\OperatorTok{@}\NormalTok{ Sigma }\OperatorTok{@}\NormalTok{ Vt}
\end{Highlighting}
\end{Shaded}

Tenga en cuenta que

\begin{Shaded}
\begin{Highlighting}[]
\NormalTok{reconstructed.shape}
\end{Highlighting}
\end{Shaded}

\begin{verbatim}
## (3, 768, 1024)
\end{verbatim}

La imagen reconstruida debe ser indistinguible de la original, excepto
por diferencias debidas a errores de punto flotante de la
reconstrucción.~Recuerde que nuestra imagen original constaba de valores
de punto flotante en el rango~.~La acumulación de error de punto
flotante de la reconstrucción puede dar como resultado valores
ligeramente fuera de este rango original:

\begin{Shaded}
\begin{Highlighting}[]
\NormalTok{reconstructed.}\BuiltInTok{min}\NormalTok{(), reconstructed.}\BuiltInTok{max}\NormalTok{()}
\end{Highlighting}
\end{Shaded}

\begin{verbatim}
## (-6.013798400233972e-15, 1.0000000000000062)
\end{verbatim}

Dado que se esperan valores en el rango, podemos utilizar clip para
eliminar el error de punto flotante:

\begin{Shaded}
\begin{Highlighting}[]
\NormalTok{reconstructed }\OperatorTok{=}\NormalTok{ np.clip(reconstructed, }\DecValTok{0}\NormalTok{, }\DecValTok{1}\NormalTok{)}
\NormalTok{plt.imshow(np.transpose(reconstructed, (}\DecValTok{1}\NormalTok{, }\DecValTok{2}\NormalTok{, }\DecValTok{0}\NormalTok{)))}
\NormalTok{plt.show()}
\end{Highlighting}
\end{Shaded}

\includegraphics{Tutorial_3_files/figure-latex/unnamed-chunk-34-11.pdf}

De hecho, realiza este recorte bajo el capó, por lo que si omite la
primera línea en la celda del código anterior, ahora, para hacer la
aproximación, debemos elegir solo los primeros k valores singulares para
cada canal de color. Esto se puede hacer usando la siguiente sintaxis

\begin{Shaded}
\begin{Highlighting}[]
\NormalTok{approx\_img }\OperatorTok{=}\NormalTok{ U }\OperatorTok{@}\NormalTok{ Sigma[..., :k] }\OperatorTok{@}\NormalTok{ Vt[..., :k, :]}
\end{Highlighting}
\end{Shaded}

Puedes ver que hemos seleccionado solo los primeros k componentes del
último eje (esto significa que hemos usado solo las primeras k columnas
de cada una de las tres matrices en la pila), y que hemos seleccionado
solo los primeros k componentes en el segundo. -al último eje de vt
(esto significa que hemos seleccionado solo las primeras k filas de cada
matriz en la pila vt y todas las columnas). Si no está familiarizado con
la sintaxis de puntos suspensivos, es un marcador de posición para otros
ejes. Ahora,

\begin{Shaded}
\begin{Highlighting}[]
\NormalTok{approx\_img.shape}
\end{Highlighting}
\end{Shaded}

\begin{verbatim}
## (3, 768, 1024)
\end{verbatim}

Que no es la forma adecuada para mostrar la imagen.~Finalmente,
reordenando los ejes a nuestra forma original de~, podemos ver nuestra
aproximación:

\begin{Shaded}
\begin{Highlighting}[]
\NormalTok{plt.imshow(np.transpose(approx\_img, (}\DecValTok{1}\NormalTok{, }\DecValTok{2}\NormalTok{, }\DecValTok{0}\NormalTok{)))}
\NormalTok{plt.show()}
\end{Highlighting}
\end{Shaded}

\includegraphics{Tutorial_3_files/figure-latex/unnamed-chunk-37-13.pdf}

Aunque la imagen no es tan nítida, utilizando una pequeña cantidad de~k
valores singulares (en comparación con el conjunto original de 768
valores), podemos recuperar muchas de las características distintivas de
esta imagen.

\hypertarget{palabras-finales}{%
\subsubsection{Palabras finales}\label{palabras-finales}}

Por supuesto, este no es el mejor método para~\emph{aproximar}~una
imagen.~Sin embargo, de hecho, hay un resultado en álgebra lineal que
dice que la aproximación que construimos anteriormente es la mejor que
podemos obtener de la matriz original en términos de la norma de la
diferencia.

\end{document}
